\begin{frame}{Mass of \chiboneThreeP in $\chib \to \Y3S \gamma$ decay (1)}
\setlength{\unitlength}{1mm}
\centering
\resizebox{0.5\textwidth}{!}{
\begin{picture}(80,60)
    %
    \put(0,0){
      \includegraphics*[width=80mm, height=60mm]{chib3s-lambda/m3p_lambda}
    }

     \put(0,15){\scriptsize \begin{sideways}Mass of \chiboneThreeP (\gevcc)\end{sideways}}
     \put(75,0){$\lambda$}

    \put(15,54){\includegraphics*[width=4mm, height=2mm]{blue}}
    \put(15,50){\includegraphics*[width=4mm, height=2mm]{red}}

    \put(20,54){\scriptsize \textcolor{blue}{$\Delta{m_{\chi_{b1,2}(3P)}} = 13\mevcc$}}
    \put(20,50){\scriptsize \textcolor{red}{$\Delta{m_{\chi_{b1,2}(3P)}} = 10\mevcc$}}
    % \put(48,140){\scriptsize \textcolor{cyan}{\sqs=7\tev (2010)}}
    
    % \put(15,169){\includegraphics*[width=4mm, height=2mm]{blue}}
    % \put(15,165){\includegraphics*[width=4mm, height=2mm]{red}}     

     % \graphpaper[5](0,0)(80, 60)
  \end{picture}
}
\begin{block}{}
\scriptsize
The mass is measured with different ratios ($\lambda$) and mass difference 
($\Delta{m_{\chi_{b1,2}(3P)}}$) between \chiboneThreeP and \chibtwoThreeP states.
The measurement is performed on the combined 2011 and 2012 datasets.
\end{block}

% \begin{alertblock}{}
% According to theory prediction, where this ratio is
% variates in range from 0.4 to 0.7, follows that \chiboneThreeP mass is in 
% the range between 10.504 and 10.514 \gevcc.
% \end{alertblock}

% In this study the mass of \chiboneOneP was fixed to \textcolor{blue}{9892} \mevcc.

\end{frame}