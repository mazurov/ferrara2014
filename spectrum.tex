\begin{frame}{Properties of quarkonium states}

{\scriptsize There is a
wide variety of quarkonium states, each differing from other by quantum
numbers: the principal quantum number ($n$), the relative angular momentum
between the quarks ($L$), the spin combination of the two quarks ($S$) and the
total angular momentum ($J$) with $J = L + S$. The spectroscopic notation
$J^{PC}$ is often used, where $P$ and $C$ are parity and
charge conjugation values, respectively. For the quarkonium states, they are
defined as $P=(-1)^{L+1}$ and $C=(-1)^{L+S}$.}

\begin{center}
\resizebox{.4\textwidth}{!}{
\begin{tabular}{lccl}\toprule
Meson & $n^{2S+1} L_J$ &  $J^{PC}$ & Mass (\mevcc)\\
\midrule
$\eta_c(1S)$    & $1^1 S_0$ & $0^{-+}$ & $2980.4 \pm 1.2$ \\
$\jpsi(1S)$    & $1^3 S_1$ & $1^{--}$ & $3096.916 \pm 0.011$ \\
$\chi_{c0}(1P)$ & $1^3 P_0$ & $0^{++}$ & $3414.75 \pm 0.31$ \\
$\chi_{c1}(1P)$ & $1^3 P_1$ & $1^{++}$ & $3510.66 \pm 0.07$ \\
$h_{c}(1P)$     & $1^3 P_1$ & $1^{++}$ & $3525.93 \pm 0.27$ \\
$\chi_{c2}(1P)$ & $1^3 P_2$ & $2^{++}$ & $3556.20 \pm 0.09$ \\
$\eta_{c}(2S)$  & $2^1 S_0$ & $0^{-+}$ & $3637 \pm 4$ \\
$\psi(2S)$      & $2^3 S_1$ & $1^{--}$ & $3686.09 \pm 0.04$ \\
\midrule
$\eta_b$        & $1^1 S_0$ & $0^{-+}$ & $9388.9 \pm 2.5 \pm 2.7$ \\
$\textcolor{red}{\Upsilon(1S)}$  & $1^3 S_1$ & $1^{--}$ & $9460.30 \pm 0.26$\\
$\textcolor{red}{\chi_{b0}(1P)}$ & $1^3 P_0$ & $0^{++}$ & $9859.44 \pm 0.42 \pm 0.31$ \\
$\textcolor{red}{\chi_{b1}(1P)}$ & $1^3 P_1$ & $1^{++}$ & $9892.78 \pm 0.26 \pm 0.31$ \\
$\textcolor{red}{\chi_{b2}(1P)}$ & $1^3 P_2$ & $2^{++}$ & $9912.21 \pm 0.26 \pm 0.31$ \\
$\textcolor{red}{\Upsilon(2S)}$  & $2^3 S_1$ & $1^{--}$ & $10023.26 \pm 0.31$ \\
$\textcolor{red}{\chi_{b0}(2P)}$ & $2^3 P_0$ & $0^{++}$ & $10232.5 \pm 0.4 \pm 0.5$ \\ 
$\textcolor{red}{\chi_{b1}(2P)}$ & $2^3 P_1$ & $1^{++}$ & $10255.46 \pm 0.22 \pm 0.5$ \\ 
$\textcolor{red}{\chi_{b2}(2P)}$ & $2^3 P_2$ & $2^{++}$ & $10268.65 \pm 0.22 \pm 0.5$ \\ 
$\textcolor{red}{\Upsilon(3S)}$  & $2^3 S_1$ & $1^{--}$ & $10355.2 \pm 0.5$ \\
\bottomrule
\end{tabular}
}
\end{center}
\end{frame}